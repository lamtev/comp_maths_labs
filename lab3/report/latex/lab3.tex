\include{settings}

\begin{document}	% начало документа

% Титульная страница
\begin{titlepage}	% начало титульной страницы

	\begin{center}		% выравнивание по центру

		\large Санкт-Петербургский Политехнический Университет Петра Великого\\
		\large Институт компьютерных наук и технологий \\
		\large Кафедра компьютерных систем и программных технологий\\[6cm]
		% название института, затем отступ 6см
		
		\huge Вычислительная математика\\[0.5cm] % название работы, затем отступ 0,5см
		\large Отчет по курсовой работе\\[0.1cm]
		\large <<Расчёт движения автомобиля>>\\[0.1cm]
		\large Задание K-3-07\\[5cm]

	\end{center}


	\begin{flushright} % выравнивание по правому краю
		\begin{minipage}{0.30\textwidth} % врезка в половину ширины текста
			\begin{flushleft} % выровнять её содержимое по левому краю

				\large\textbf{Работу выполнил:}\\
				\large Ламтев А.Ю.\\
				\large {Группа:} 23501/4\\
				
				\large \textbf{Преподаватель:}\\
				\large к.т.н., доц. Цыган В.Н.

			\end{flushleft}
		\end{minipage}
	\end{flushright}
	
	\vfill % заполнить всё доступное ниже пространство

	\begin{center}
	\large Санкт-Петербург\\
	\large \the\year % вывести дату
	\end{center} % закончить выравнивание по центру

\thispagestyle{empty} % не нумеровать страницу
\end{titlepage} % конец титульной страницы

\vfill % заполнить всё доступное ниже пространство


\section{Цель работы}
Численно решить дифференциальное уравнение 2-го порядка.

\section{Решаемые задачи}
\begin{enumerate}

\item Привести дифференциальное уравнение второго порядка

\label{equation:1}
\begin{equation}
	t(t+1)y'' + (3t+2)y'  + y = 0
\end{equation}

к системе двух дифференциальных уравнений первого порядка.

	Начальные условия: $\left.y \right|_{t=1} = 1$, $\left.y' \right|_{t=1} = -1$, $t \in[1, 2]$  

    Точное решение: $y(t)=\frac{1}{t}$	

\item Решить с шагом $h = 0.1$

\begin{enumerate}[label=\arabic*)]
\item Используя RKF45
\item Используя метод Эйлера-Коши
\end{enumerate}

\item Сравнить результаты полученные двумя методами с точным решением.

\end{enumerate}

\section{Ход выполнения работы}

Вначале дифференциальное уравнение \ref{equation:1} при помощи подстановок $x_1(t) = y(t)$ и $x_2(t) = y'(t)$ было приведено к системе:

\begin{displaymath}
\begin{cases}
	x'_1(t) = 0 \cdot x_1(t) + 1 \cdot x_2(t)
	\\
	x'_2(t) = - \frac{1}{t(t+1)} \cdot x_1(t) - \frac{3t+2}{t(t+1)} \cdot x_2(t)
\end{cases}
\end{displaymath}

Затем было разработано программное обеспечение на языке программирования \textbf{java}, позволяющее решить поставленные задачи. Данное программное обеспечение включает в себя:
\begin{itemize}

\item Библиотеку \textbf{Diffeqs}.

 Она включает в себя функцию \textbf{rkf45}, которая содержит вызов стандартной форсайтовской функции \textbf{rkf45}, разработанной на языке программирования \textbf{с}.

 Еще \textbf{Diffeqs} содержит самостоятельно написанную функцию \textbf{eulerCauchy}. Данная функция численно решает дифференциальное уравнение, используя метод Эйлера-Коши 2-й степени:
 
\begin{displaymath}
\begin{cases}
	x_{n+1}^{*} = x_n + hf(t_n, x_n)
	\\
	x_{n+1} = x_n + \frac{h}{2} \Big (f(t_n, x_n) + f(t_{n+1}, x_{n+1}^{*}) \Big) 
\end{cases}
\end{displaymath} 
 
 Функция \textbf{eulerCauchy} представлена в листинге \ref{code:EulerCauchy}.

\item Приложение, решающее решающее заданную систему дифференциальных уравнений точно (по формуле $y(t) = \frac{1}{t}$), при помощи \textbf{rkf45} и функцией \textbf{eulerCauchy}, реализующей метод Эйлера-Коши (листинг \ref{code:Lab3}).

\end{itemize}

На рисунках \ref{pic:demo1} и \ref{pic:demo2} изображен вывод работы программы: результат решения системы дифференциальных уравнений на отрезке $1\dots2$ с шагом $h=0.1$ стандартной функцией \textbf{rkf45}, функцией \textbf{eulerCauchy} и по аналитически полученной точной формуле соответственно.

\begin{figure}[H]
    \centering
    \begin{subfigure}[t]{0.5\textwidth}
        \centering
        \includegraphics[height=1.72in]{rkf45}
        \caption{RKF45}
    \end{subfigure}%
    ~ 
    \begin{subfigure}[t]{0.5\textwidth}
        \centering
        \includegraphics[height=1.72in]{euler_cauchy}
        \caption{Метод Эйлера-Коши}
    \end{subfigure}
    \caption{Вывод программы}
    \label{pic:demo1}
\end{figure}

\begin{figure}[H]
    \centering
    \includegraphics[height=2.2in]{exact}
    \caption{Вывод программы: точное решение}
    \label{pic:demo2}
\end{figure} 

По выводу программы видно, что значения, полученные с использованием функции \textbf{rkf45}, совпадают с точными значениями с точностью 7 - 8 знаков после точки, а значения, полученные методом Эйлера-Коши имеют точность 1 - 2 знака после точки. 

\section{Выводы}

Решение системы дифференциальных уравнений при помощи функции \textbf{rkf45} оказалось более точным, чем решение методом Эйлера-Коши. Полученный результат был ожидаем, потому что функция \textbf{rkf45} реализует методы Рунге-Кутты 4-й и 5-й степени, в то время как метод Эйлера-Коши имеет лишь 2-ую степень.

\newpage

\section*{Приложение 1. Листинги кода}

\captionof{lstlisting}{Приложение}
\lstinputlisting[label=code:Lab3, linerange={15-87}]{../../../app/src/main/java/com/lamtev/comp_maths_labs/lab3/app/Lab3.java}
\parindent=1cm

\captionof{lstlisting}{Функция eulerCauchy}
\lstinputlisting[label=code:EulerCauchy, linerange={7-39}]{../../../diffeqs_lib/src/main/java/com/lamtev/comp_maths_labs/lab3/diffeqs_lib/EulerCauchy.java}
\parindent=1cm

\end{document}
