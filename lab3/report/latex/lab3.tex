\include{settings}

\begin{document}	% начало документа

% Титульная страница
\begin{titlepage}	% начало титульной страницы

	\begin{center}		% выравнивание по центру

		\large Санкт-Петербургский Политехнический Университет Петра Великого\\
		\large Институт компьютерных наук и технологий \\
		\large Кафедра компьютерных систем и программных технологий\\[6cm]
		% название института, затем отступ 6см
		
		\huge Вычислительная математика\\[0.5cm] % название работы, затем отступ 0,5см
		\large Отчет по курсовой работе\\[0.1cm]
		\large <<Расчёт движения автомобиля>>\\[0.1cm]
		\large Задание K-3-07\\[5cm]

	\end{center}


	\begin{flushright} % выравнивание по правому краю
		\begin{minipage}{0.30\textwidth} % врезка в половину ширины текста
			\begin{flushleft} % выровнять её содержимое по левому краю

				\large\textbf{Работу выполнил:}\\
				\large Ламтев А.Ю.\\
				\large {Группа:} 23501/4\\
				
				\large \textbf{Преподаватель:}\\
				\large к.т.н., доц. Цыган В.Н.

			\end{flushleft}
		\end{minipage}
	\end{flushright}
	
	\vfill % заполнить всё доступное ниже пространство

	\begin{center}
	\large Санкт-Петербург\\
	\large \the\year % вывести дату
	\end{center} % закончить выравнивание по центру

\thispagestyle{empty} % не нумеровать страницу
\end{titlepage} % конец титульной страницы

\vfill % заполнить всё доступное ниже пространство


\section{Цель работы}
Исследовать зависимость нормы матрицы от возмущения исходных данных.

\section{Решаемые задачи}
\begin{enumerate}

\item Составить процедуру вычисления по заданной матрице $A_{N \times N}$ матрицы \\$R = A^{-1} A - E$ и её нормы $||R|| = \max_{k} \sum_j^N |R_{jk}|$.

\item Построить матрицы A при $x_k = \frac{1 + cos(k)}{sin^2(k)}, k = 1, \dots, 4$ и $x_5 = \frac{1+cos(1)}{sin^2(1 + \varepsilon)}$, для значений $\varepsilon = 10^{-1}, 10^{-2}, 10^{-3}, \dots, 10^{-16}, 0$ и $N = 5$.

\begin{displaymath}
A=
  \begin{pmatrix}
    1 & 1 & \dots & 1 \\
    x_1 & x_2 & \dots & x_N \\
    \dots & \dots & \dots & \dots & \\
    x_1^{N-1} & x_2^{N-1} & \dots & x_N^{N-1} \\
  \end{pmatrix}
\end{displaymath}


\item  Исследовать зависимость погрешности вычисления $||R||$ от $\varepsilon$.
\end{enumerate}


\section{Ход выполнения работы}

В ходе выполнения работы было разработано программное обеспечение на языке программирования \textbf{java}, позволяющее решить поставленные задачи. Данное программное обеспечение включает в себя:
\begin{itemize}

\item Библиотеку \textbf{MatrixUtil} с функциями \textbf{decomp} и \textbf{solve}, которые содержат вызовы стандартных форсайтовских функций \textbf{decomp} и \textbf{solve}, разработанных на языке программирования \textbf{с}, из динамической библиотеки.

\item Библиотеку \textbf{Matrix}, позволяющую совершать различные действия над матрицами, в том числе, \textit{обращение, произведение, вычитание, вычисление нормы}, которые необходимы для решения поставленных задач.

 \textit{Обращение} матрицы реализовано как n итераций решения системы $\textbf{A}x = \textbf{B}$ при помощи функций \textbf{decomp} и \textbf{solve} c $\textbf{B}$ -- i-м столбцом единичной матрицы на i-й итерации и n -- порядком матрицы. Реализация представлена в листинге \ref{code:Matrix:inverted}.
 
 Реализации \textit{произведения, вычитания, вычисления нормы} представлены в листингах \ref{code:Matrix:multiply}, \ref{code:Matrix:minus} и \ref{code:Matrix:calculateNormAsMaximumAbsoluteColumnSum} соответственно.

\item Приложение, решающее поставленные задачи и использующее ранее перечисленные библиотеки (листинг \ref{code:Lab2}).

\end{itemize} 

\begin{figure}[H]
    \centering
    \begin{subfigure}[t]{0.5\textwidth}
        \centering
        \includegraphics[height=1.75in]{rkf45}
        \caption{RKF45}
    \end{subfigure}%
    ~ 
    \begin{subfigure}[t]{0.5\textwidth}
        \centering
        \includegraphics[height=1.75in]{euler_cauchy}
        \caption{Метод Эйлера-Коши}
    \end{subfigure}
    \caption{Вывод программы}
    \label{pic:demo1}
\end{figure} 

На рисунках \ref{pic:graphic1} и \ref{pic:graphic2} изображены графики зависимости нормы матрицы от $\varepsilon$ на всем интервале и на отрезке $[10^{-16}, 10^{-1}]$ соответственно.

\section{Выводы}

\newpage

\section*{Приложение 1. Листинги кода}

\end{document}
