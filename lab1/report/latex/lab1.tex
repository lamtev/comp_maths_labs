\include{settings}

\begin{document}	% начало документа

% Титульная страница
\begin{titlepage}	% начало титульной страницы

	\begin{center}		% выравнивание по центру

		\large Санкт-Петербургский Политехнический Университет Петра Великого\\
		\large Институт компьютерных наук и технологий \\
		\large Кафедра компьютерных систем и программных технологий\\[6cm]
		% название института, затем отступ 6см
		
		\huge Вычислительная математика\\[0.5cm] % название работы, затем отступ 0,5см
		\large Отчет по курсовой работе\\[0.1cm]
		\large <<Расчёт движения автомобиля>>\\[0.1cm]
		\large Задание K-3-07\\[5cm]

	\end{center}


	\begin{flushright} % выравнивание по правому краю
		\begin{minipage}{0.30\textwidth} % врезка в половину ширины текста
			\begin{flushleft} % выровнять её содержимое по левому краю

				\large\textbf{Работу выполнил:}\\
				\large Ламтев А.Ю.\\
				\large {Группа:} 23501/4\\
				
				\large \textbf{Преподаватель:}\\
				\large к.т.н., доц. Цыган В.Н.

			\end{flushleft}
		\end{minipage}
	\end{flushright}
	
	\vfill % заполнить всё доступное ниже пространство

	\begin{center}
	\large Санкт-Петербург\\
	\large \the\year % вывести дату
	\end{center} % закончить выравнивание по центру

\thispagestyle{empty} % не нумеровать страницу
\end{titlepage} % конец титульной страницы

\vfill % заполнить всё доступное ниже пространство


\section{Цель работы}
Сравнить точность интерполяционных сплайн-полинома и полинома Лагранжа для заданной функции.

\section{Решаемые задачи}
\begin{enumerate}

\item Для $2 \leq x \leq 3$ с шагом $h = 0.1$ вычислить значение функции f(x) с использованием программы \textbf{QUANC8}, где $f(x) = \int_{0}^{x} \frac{sin(t)}{t} dt$.

\item По полученным точкам построить сплайн-функцию и полином Лагранжа 10-й степени.

\item  В точках $x_k = (k - 0.5) \cdot h + 2$, для $k = 1, 2, \dots, 10$ сравнить значение сплайн-функции и полинома с точным значением f(x), которое вычисляется при помощи \textbf{QUANC8} с высокой точностью.
\end{enumerate}


\section{Ход выполнения работы}

В ходе выполнения работы было разработано программное обеспечение, позволяющее решить поставленные задачи. Исходный код программы представлен в приложении 1.

 На рис. \ref{pic:demo} представлен вывод разработанной программы. В столбце \textbf{x\_k} находятся точки, в которых были посчитаны значения интерполяционного полинома Лагранжа (столбец \textbf{lagrange}), интерполяционного сплайн-полинома (столбец \textbf{spline}) и таблично заданной функции (столбец \textbf{quanc8}).

\begin{figure}[H]
	\begin{center}
		\includegraphics[scale=1]{demo}
		\caption{Результат работы программы} 
		\label{pic:demo} % название для ссылок внутри кода
	\end{center}
\end{figure}

Как видно по выводу программы, значения полинома Лагранжа совпадают с точным значением функции в заданных точках, а значения сплайн-полинома отличаются от точных значений функции в заданных точках.

\section{Выводы}
Из сравнения значений заданной функции со значениями интерполяционных полиномов в определённом наборе точек следует, что интерполяционный полином Лагранжа больше подходит для интерполяции заданной функции, если критерием близости является равенство именно в конкретном определённом наборе точек. Если критерий близости другой, то из полученных данных нельзя сделать вывод о преимуществе одного интерполяционного полинома над другим.

\newpage

\section*{Приложение 1. Листинги кода} \label{sec:appendix1}

\lstinputlisting[
	label=code:main,
	caption={main.cpp},% для печати символ '_' требует выходной символ '\'
]{../../../app/src/main.cpp}
\parindent=1cm % командна \lstinputlisting сбивает параментры отступа

\lstinputlisting[
	label=code:util,
	caption={util.cpp},% для печати символ '_' требует выходной символ '\'
]{../../../lib/src/util.cpp}
\parindent=1cm % командна \lstinputlisting сбивает параментры отступа

\lstinputlisting[
	label=code:quanc8calculation,
	caption={quanc8\_calculation.cpp},% для печати символ '_' требует выходной символ '\'
]{../../../lib/src/quanc8_calculation.cpp}
\parindent=1cm % командна \lstinputlisting сбивает параментры отступа

\lstinputlisting[
	label=code:lagrangecalculation,
	caption={lagrange\_calculation.cpp},% для печати символ '_' требует выходной символ '\'
]{../../../lib/src/lagrange_calculation.cpp}
\parindent=1cm % командна \lstinputlisting сбивает параментры отступа

\lstinputlisting[
	label=code:splinecalculation,
	caption={spline\_calculation.cpp},% для печати символ '_' требует выходной символ '\'
]{../../../lib/src/spline_calculation.cpp}
\parindent=1cm % командна \lstinputlisting сбивает параментры отступа

\end{document}
