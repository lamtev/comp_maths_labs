\include{settings}

\begin{document}	% начало документа

% Титульная страница
\begin{titlepage}	% начало титульной страницы

	\begin{center}		% выравнивание по центру

		\large Санкт-Петербургский Политехнический Университет Петра Великого\\
		\large Институт компьютерных наук и технологий \\
		\large Кафедра компьютерных систем и программных технологий\\[6cm]
		% название института, затем отступ 6см
		
		\huge Вычислительная математика\\[0.5cm] % название работы, затем отступ 0,5см
		\large Отчет по курсовой работе\\[0.1cm]
		\large <<Расчёт движения автомобиля>>\\[0.1cm]
		\large Задание K-3-07\\[5cm]

	\end{center}


	\begin{flushright} % выравнивание по правому краю
		\begin{minipage}{0.30\textwidth} % врезка в половину ширины текста
			\begin{flushleft} % выровнять её содержимое по левому краю

				\large\textbf{Работу выполнил:}\\
				\large Ламтев А.Ю.\\
				\large {Группа:} 23501/4\\
				
				\large \textbf{Преподаватель:}\\
				\large к.т.н., доц. Цыган В.Н.

			\end{flushleft}
		\end{minipage}
	\end{flushright}
	
	\vfill % заполнить всё доступное ниже пространство

	\begin{center}
	\large Санкт-Петербург\\
	\large \the\year % вывести дату
	\end{center} % закончить выравнивание по центру

\thispagestyle{empty} % не нумеровать страницу
\end{titlepage} % конец титульной страницы

\vfill % заполнить всё доступное ниже пространство


\section{Цель работы}
Сравнить точность интерполяционного полинома Лагранжа и интерполяционного сплайн-полинома для заданной функции.

\section{Решаемые задачи}
\begin{enumerate}

\item Для $2 \leq x \leq 3$ с шагом $h = 0.1$ вычислить значение функции f(x) с использованием программы \textbf{QUANC8}, где $f(x) = \int_{0}^{x} \frac{sin(t)}{t} dt$.

\item По полученным точкам построить сплайн-функцию и полином Лагранжа 10-й степени.

\item  В точках $x_k = (k - 0.5) \cdot h + 2$, для $k = 1, 2, \dots, 10$ сравнить значение сплайн-функции и полинома с точным значением f(x), которое вычисляется программой \textbf{QUANC8} с заданием высокой точности.
\end{enumerate}


\section{Ход выполнения работы}

В ходе выполнения работы было разработано программное обеспечение на языке программирования \textbf{с++}, позволяющее решить поставленные задачи. При его разработке использовались стандартные функции \textbf{QUANC8}, \textbf{SPLINE} и \textbf{SEVAL}. Исходный код представлен в приложении 1.

 На рисунках \ref{pic:demo1}, \ref{pic:demo2}  и \ref{pic:demo3} изображён вывод разработанной программы с абсолютной погрешностью при вычислении интеграла по \textbf{QUANC8} $10^{-7}$, $10^{-13}$ и $10^{-19}$ соответственно. 
 
\begin{figure}[H]
	\begin{center}
		\includegraphics[scale=0.70]{table_10_7}
		\caption{Вывод программы с абсолютной погрешностью $10^{-7}$}
		\label{pic:demo1} % название для ссылок внутри кода
	\end{center}
\end{figure}

\begin{figure}[H]
	\begin{center}
		\includegraphics[scale=0.70]{table_10_13}
		\caption{Вывод программы с абсолютной погрешностью $10^{-13}$} 
		\label{pic:demo2} % название для ссылок внутри кода
	\end{center}
\end{figure}

\begin{figure}[H]
	\begin{center}
		\includegraphics[scale=0.70]{table_10_19}
		\caption{Вывод программы с абсолютной погрешностью $10^{-19}$} 
		\label{pic:demo3} % название для ссылок внутри кода
	\end{center}
\end{figure} 
 
  Над каждой таблицей указана установленная абсолютная погрешность вычисления интеграла по \textbf{QUANC8}. В столбце \textbf{x\_k} находятся точки, в которых были посчитаны значения интерполяционного полинома Лагранжа (столбец \textbf{lagrange}), значения интерполяционного сплайн-полинома (столбец \textbf{spline}) и точные значения (по \textbf{QUANC8} с высокой точностью) таблично-заданной функции (столбец \textbf{exact value by QUANC8}). Значения выведены с 16-ю разрядами после точки. 
 
 Экспериментально было определено, что установка абсолютной погрешности ниже чем $10^{-19}$ не имеет смысла, потому что реальная абсолютная погрешность, вычисляемая программно, остаётся больше, чем $10^{-19}$. Это отражено на рисунке \ref{pic:demo4}.

\begin{figure}[H]
\begin{center}
	\begin{subfigure}[b]{0.24\textwidth}
		\includegraphics[scale=0.46]{quanc8_10_7}
		\caption{\\Для абсолютной \\погрешности $10^{-7}$}
		\label{pic:demo4:1}
	\end{subfigure}
	\begin{subfigure}[b]{0.24\textwidth}
		\includegraphics[scale=0.46]{quanc8_10_13}
		\caption{\\Для абсолютной \\погрешности $10^{-13}$}
		\label{pic:demo4:2}
	\end{subfigure}
	\begin{subfigure}[b]{0.24\textwidth}
		\includegraphics[scale=0.46]{quanc8_10_19}
		\captionsetup{justification=centering}
		\caption{\\Для абсолютной \\погрешности $10^{-19}$}
		\label{pic:demo4:3}
	\end{subfigure}
	\begin{subfigure}[b]{0.24\textwidth}
		\includegraphics[scale=0.46]{quanc8_10_20}
		\captionsetup{justification=centering}
		\caption{\\Для абсолютной \\погрешности $10^{-20}$}
		\label{pic:demo4:4}
	\end{subfigure}
	\caption{Вывод информации о вычислениях функцией QUANC8}
	\label{pic:demo4}
\end{center}
\end{figure}

\section{Выводы}

По результату работы программы можно заметить, что для абсолютной погрешности, равной $10^{-7}$, $10^{-13}$ и $10^{-19}$, точность вычисления значений полинома Лагранжа равна от 14 до 16 знаков после точки на всём интервале, а точность вычисления значений сплайн-полинома равна от 3 - 4 знаков после точки по краям интервала до 6 знаков после точки в середине интервала. Таким образом, интерполяционный полином Лагранжа больше подходит для интерполяции заданной функции, если критерием близости является равенство именно в конкретном определённом наборе точек.% Если критерий близости другой, то из полученных данных нельзя сделать вывод о преимуществе одного интерполяционного полинома над другим.

\newpage

\section*{Приложение 1. Листинги кода}

\end{document}
